In this paper, we first aimed to reconstruct the model described in the Unirep paper\cite{unirep}. This went rather well, considering the significantly less hardware and training time available. Our best model scores a Spearman correlation within $0.11$ of the results described in the Tape paper\cite{tape} and scores better by $0.03$ compared to the values they reported in the Unirep paper. Our LSTM models also found good separation between the different structural classifications of the proteins in the Scope dataset.\cite{scope} Our attempt to create a CNN that could replicate these results did not go as well. It was only able to get a $0.351$ Spearman correlation, and the structural classification plots did not have significant boundaries between the classes. This may be due to the simple model we utilized.\\

\noindent
While writing this project we have become familiar with the fundamental of deep learning, and have learned a lot about the inner workings of both RNNs and CNNs. In the making of these models, we also got familiar with the basic terminology regarding proteins and amino acids. We gained this knowledge through both reading the papers cited throughout, and through practical experience, primarily using PyTorch to implement our models.