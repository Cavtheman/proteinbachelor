In this section we will go over the results of the various experiments we have concocted. The results will be evaluated in two separate ways. For each experiment there are two models, the "final" model that was saved after all training was finished, and the "minloss" model which was saved during training when the model achieved the lowest loss. With the last experiment being the exception, all results shown are from the "final" models.\\

\noindent
The first evaluation will be on a structural classification dataset, which will be a qualitative analysis in which we perform t-distributed stochastic neighbor embedding (TSNE) dimensionality reduction on the data, and see whether it is able to cleanly separate the different types of protein structures. TSNE is preferred here over something like principal component analysis (PCA) due to this non-linearity of the reduction. This evaluation will be performed on the structural classification dataset. While using a classification method like k-nearest neighbors (KNN) might seem intuitive to quantitatively evaluate this, the non-linearity means that neither distance nor density is preserved between data points. We will also look at the next token prediction accuracy for the LSTM, but because we want to quantify how well it represents the proteins, this is not necessarily a good way to do that. It will mostly be used for comparison to see if there are any interesting insights.\\

\noindent
The second evaluation will be on the stability dataset. This evaluation will be quantitative, using Spearman's rank correlation coefficent. This coefficient measures the degree to which the relationship between two inputs can be described using a monotonic function. This means that having a high spearman correlation is equivalent to having a highly descriptive representation. It does not necessarily mean that the linear regression model we have trained provides good results in and of itself, since the coefficient does not describe a 1:1 correlation. This can be seen in figure ~\ref{fig:spearman}.

\begin{figure}[!ht]
  \centering
  \includegraphics[width=0.4\linewidth]{latex/imgs/spearman_fig.png}
  \caption{Graph showing that high spearman correlation does not necessarily mean a good score. Image source:\cite{spearman}}\label{fig:spearman}
\end{figure}

\subsection{LSTM experiments}

\subsubsection{Layers vs no layers}
Stacking LSTMs is a way to increase the non-linearity of the network. The idea is similar to increasing the depth of conventional neural networks. In a stacked LSTM, there are two or LSTM cells per element in the input sequence. The way this works is that the hidden state of the first cell is passed along as the input to the next cell and so on. For this experiment we have trained two models with the following identical hyperparameters:
\begin{itemize}
    \item Character embedding size: 30
    \item Starting learning rate: $8e-4$
    \item Learning rate schedule: Multiply by $0.2$ every 5 epochs
    \item Hidden layer size: $512$
    \item Training time: 30 epochs
\end{itemize}
The only difference between these models is that one has two layers, and the other has only one.\\

\noindent
We decided to perform this experiment because increasing the non-linearity of simpler networks has been shown to make them significantly easier to train. Since proteins have such complicated structures though increasing the non-linearity might help the network learn this structure better.\\

\noindent
While adding layers to the network does increase the non-linearity it also makes the training significantly slower. Because each layer is essentially an entire LSTM in and of itself, going from one to two layers actually doubles the training time for the same amount of epochs. We hypothesize that due to the increased non-linearity, the network with two layers will learn a better representation of the protein structure.

\subsubsection{Dropout vs no dropout}
\begin{figure}[!ht]
  \centering
  % Accuracy on test set: 13.03%, loss 2.83
  \includegraphics[width=0.49\linewidth]{latex/imgs/tsne_2_layer_no_drop_final.png}
  % Accuracy on test set: 13.02%, loss 2.82
  \includegraphics[width=0.49\linewidth]{latex/imgs/tsne_2_layer_05_drop_final.png}
  % 0.400 and 0.427
  \includegraphics[width=0.49\linewidth]{latex/imgs/spearman_2_layer_no_drop_final.png}
  % 0.518 and 0.539
  \includegraphics[width=0.49\linewidth]{latex/imgs/spearman_2_layer_05_drop_final.png}
  \caption{TSNE dimensionality reduction  and Spearman's rho data plots of the two models. Left is without dropout, right is with $50\%$ dropout.}
\end{figure}

\begin{table}[!ht]
\begin{tabular}{|l|l|l|l|}
\hline
             & Next token prediction accuracy & Test Loss & Spearman's rho\\ \hline
No dropout   & 13.03\%                        & 2.83      & 0.400         \\ \hline
50\% dropout & 13.02\%                        & 2.82      & 0.518         \\ \hline
\end{tabular}
\end{table}

\subsubsection{Feature size}
The feature size is the hyperparameter which controls the size of the internal layers of the LSTM. Increasing this should mean that the network can learn more complex functions. However, it also increases training time in a linear way. The increase is linear because for every feature you add, you have to backpropagate through one more. For this experiment we've trained three models with the following identical hyperparameters:
\begin{itemize}
	\item Layers: $1$
	\item Character embedding size: 30
	\item Starting learning rate: $8e-4$
	\item Learning rate schedule: Multiply by $0.2$ every 5 epochs
	\item Training time: 6 hours
\end{itemize}
The only difference between the models is the hidden layer size or feature size. We trained three models, one with $256$ features, one with $512$ features and one with $1024$ features. Since training time is the most sparse resource for us, we decided that we would compare the results for equal real-time training time, not the number of epochs as with the other experiments.\\

\noindent
For this experiment we expect there to be some middle ground between the size of the model and how well it learns the representation. Since protein structures are very complex, we hypothesise that the larger models will perform better, even though they are not able to train for as many epochs.

\subsubsection{Learning Rate}
The learning rate of any kind of neural network is an incredibly important hyperparameter. For this experiments we have trained two models with the following identical hyperparameters:
\begin{itemize}
	\item Layers: $1$
	\item Character embedding size: 30
	\item Starting learning rate: $8e-4$
	\item Hidden layer size: $512$
	\item Training time: 30 epochs
\end{itemize}
The only difference between the networks was that one of them had a learning rate schedule. This schedule meant that every $5$ epochs, the learning rate would be multiplied by $0.2$. \\

\noindent
We decided to lay out this test because we noticed that when training, the training loss would vary significantly as training went on, and that this variance increased significantly when training for long periods of time. We thought that reducing the learning rate could potentially alleviate this problem slightly. A learning rate schedule is a good way to do this, because simply lowering the learning rate significantly at the beginning would mean that the model would take a very long time to reach the point where it becomes helpful.\\

\noindent
We hypothesise that the reason the loss is fluctuating so much is because of a learning rate that is to high, effectively "jumping over" a local minimum. Thus, decreasing the learning rate as training goes on should cause the model to converge to a better solution faster.

\subsubsection{Minimum loss model vs last model} % Finished
%For this experiment the Spearman correlation and TSNE plots can be found in the appendices.
\begin{table}[!ht]
  \centering
\begin{tabular}{|l|lll|}
\hline
LSTM Results                          &                                                     & Final Models                       &                \\ \cline{2-4}
                                      & \multicolumn{1}{l|}{Next token prediction}          & \multicolumn{1}{l|}{Test Loss}     & Spearman's rho \\ \hline
2-layer, 50\% dropout, 512 features   & \multicolumn{1}{l|}{13.02\%}                        & \multicolumn{1}{l|}{2.82}          & 0.518          \\ \hline
2-layer, no dropout, 512 features     & \multicolumn{1}{l|}{13.03\%}                        & \multicolumn{1}{l|}{2.83}          & 0.400          \\ \hline
1-layer, no lr schedule, 512 features & \multicolumn{1}{l|}{13.34\%}                        & \multicolumn{1}{l|}{2.83}          & \textbf{0.605} \\ \hline
1-layer, 256 features                 & \multicolumn{1}{l|}{11.41\%}                        & \multicolumn{1}{l|}{2.86}          & 0.435          \\ \hline
1-layer, 512 features                 & \multicolumn{1}{l|}{12.43\%}                        & \multicolumn{1}{l|}{2.84}          & 0.428          \\ \hline
1-layer, 1024 features                & \multicolumn{1}{l|}{\textbf{14.37\%}}               & \multicolumn{1}{l|}{\textbf{2.79}} & 0.507          \\ \hline
\end{tabular}
\caption{Results for final models}
\label{tab:final}
\end{table}
\begin{table}[!ht]
  \centering
\begin{tabular}{|l|lll|}
\hline
LSTM Results                          &                                                     & Minloss models                     &                \\ \cline{2-4}
                                      & \multicolumn{1}{l|}{Next token prediction}          & \multicolumn{1}{l|}{Test Loss}     & Spearman's rho \\ \hline
2-layer, 50\% dropout, 512 features   & \multicolumn{1}{l|}{13.03\%}                        & \multicolumn{1}{l|}{2.82}          & 0.539          \\ \hline
2-layer, no dropout, 512 features     & \multicolumn{1}{l|}{13.78\%}                        & \multicolumn{1}{l|}{2.81}          & 0.427          \\ \hline
1-layer, no lr schedule, 512 features & \multicolumn{1}{l|}{13.36\%}                        & \multicolumn{1}{l|}{2.83}          & 0.592          \\ \hline
1-layer, 256 features                 & \multicolumn{1}{l|}{11.40\%}                        & \multicolumn{1}{l|}{2.87}          & 0.424          \\ \hline
1-layer, 512 features                 & \multicolumn{1}{l|}{12.43\%}                        & \multicolumn{1}{l|}{2.85}          & 0.414          \\ \hline
1-layer, 1024 features                & \multicolumn{1}{l|}{\textbf{14.22\%}}               & \multicolumn{1}{l|}{\textbf{2.80}} & \textbf{0.627} \\ \hline
\end{tabular}
\caption{Results for minloss models}
\label{tab:minloss}
\end{table}


\subsubsection{CNN}
